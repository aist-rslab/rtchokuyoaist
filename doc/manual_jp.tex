\documentclass[a4paper,10pt]{article}

\usepackage{CJKutf8}
% \usepackage[UKenglish]{babel}
\usepackage[overlap, CJK]{ruby}
\usepackage{CJKulem}

\renewcommand{\rubysep}{-0.2ex}
\newenvironment{Japanese}{%
  \CJKfamily{min}%
  \CJKtilde
  \CJKnospace}{}

\usepackage{listings}
\usepackage{a4wide}
\usepackage{tabularx}
\usepackage{booktabs}

\lstloadlanguages{Bash}
\lstset{basicstyle=\small\ttfamily, escapeinside={(*@}{@*)},
showstringspaces=false, breaklines=true, breakatwhitespace=true,
tabsize=4, language=Bash, numberfirstline=true,
commentstyle=\itshape\color{CommentGreen},
stringstyle=\color{DataTypeBlue}}
\newcommand{\ilcode}[1]{\lstinline|#1|}

\title{RTC:HokuyoAist}
\author{Geoffrey Biggs (ジェフ・ビグズ)\\
geoffrey.biggs@aist.go.jp}

\begin{document}
\begin{CJK}{UTF8}{}
\begin{Japanese}

\maketitle

\section{Introduction}
\label{sec:intro}

RTC:HokuyoAistはOpenRTM-aist用のRTコンポーネントです。北洋電気のレーザーセンサーのドライバを
提供します。Gearbox\footnote{http://gearbox.sourceforge.net/}からのhokuyo\_aistライブラリをラップします。URG-04LX(Classic-URG)、
UGH-08LX(Hi-URG)、UTM-30LX(Top-URG)、UXM-30LX(Tough-URG)及びUXM-30LX-Eを
含むレーザーセンサーの、すべての現在のモデルで機能します。

コンポーネントが活性化されるまで、レーザーは起動しません。その後、コンポーネントがプルモードである
ようには構成されなければ、レーザーは起動します。機動時、レーザーの内部クロックがコミュニケーション
遅れを含むコンピューターのクロックに測定されます。データのタイムスタンプはこの測定された時間を
使用して計算されます。したがって、測定はデータのユーザによって必要ではありません。

このソフトウェアは産業技術総合研究所で開発されています。承認番号はH22PRO-1194です。
開発は新エネルギー・産業技術総合開発機構(Project for Strategic Development of Advanced
Robotics Elemental Technologies)に支えられました。このソフトウェアは
Eclipse Public License -v 1.0 (EPL)でライセンスされています。LICENSE.txtを
参照してください。

\section{条件}
\label{sec:requirements}

RTC:HokuyoAistはGearboxからのhokuyo\_aistライブラリを使います。バージョン2.0.0以上が必要
です\footnote{Gearbox 10.11にあります。}。
このライブラリはWindows用のインストーラに含まれています。

RTC:HokuyoAistははOpenRTM-aistのC++版が必要です。

RTC:StageはCMake\footnote{http://www.cmake.org/}を使います。Cmake
2.6以上は必要です。

RTC:HokuyoAistはWindows、Linux及びMacOS Xに使うことができます。レーザーとの
コミュニケーションはGearboxの「flexiport」というライブラリを使います。このライブラリを
インストールしなければなりません。このライブラリはWindows用のインストーラに含まれています。

\section{インストール}
\label{sec:installation}

\subsection{インストーラ}

Windowsのユーザはインストーラペケージを使用してコンポーネントをインストールすることができます。
これはコンポーネント及びそのすべての必要なライブラリをインストールします。Windowsでインストール
の推奨の方法です。

\begin{enumerate}
  \item インストーラをダウンロードしてください。
  \item インストールを始めるためにインストーラをダブルクリックしてください。
  \item コンポーネントをインストールする指示にしたがってください。
  \item 環境変数変更がコンポーネントを使用する前に効果を現わすためにコンピューターを再起動する
必要があるかもしれません。
\end{enumerate}

RTC:HokuyoAistは\verb|rtchokuyoaist_standalone|の実行をダブルクリックして実行することが
できます。あるいは、\verb|rtc_hokuyoaist|は初期化関数の\verb|rtc_init|を利用して、
マネージャにロードすることができます。

\subsection{ソースから}

ソースを使う場合は以下の手続きでインストールしてください。

\begin{enumerate}
  \item ソースをダウンロードして解凍してください。

  \verb|tar -xvzf rtchokuyoaist-2.0.0.tar.gz|
  \item 解凍されたフォルダに入ってください。

  \verb|cd rtchokuyoaist-2.0.0|
  \item ``build''フォルダを作ってください。

  \verb|mkdir build|
  \item ``build''フォルダに入ってください。

  \verb|cd build|
  \item CMakeを実行してください。

  \verb|cmake ../|
  \item エラーはなかったら、makeを実行してください。

  \verb|make|
  \item make
  installでコンポーネントをインストールしてください。選択されたインストール
  場所の許可はあるかを確認してください。

  \verb|make install|
  \item
  インストールする場所はccmakeを実効して\verb|CMAKE_INSTALL_PREFIX|を
  設定したら変更することは可能です。

  \verb|ccmake ../|
\end{enumerate}

コンポーネントを使える用になりました。コンフィグレーションは次のセクションに
参照してください。

RTC:は\verb|rtchokuyoaist_standalone|の実行(\verb|${prefix}/bin|に
インストールされました)によりスタンドアローンモードで実行 することができます。
あるいは、\verb|librtchokuyoaist.so|は初期化関数の\verb|rtc_init|を利用して、
マネージャにロードすることができます。このライブラリは\verb|${prefix}/lib|
または\verb|${prefix}/lib64|にインストールされます。

\section{コンフィグレーション}
\label{sec:configuration}

使えるコンフィグレーションパラメータはテーブル~\ref{tab:config_params}に参照
してください。

\begin{table}[t]
  \centering
  \begin{tabularx}{\columnwidth}{lX}
    \toprule
    パラメータ & 意味 \\
    \midrule
    port\_opts & レーザーにポートを開くために使用されるオプションを変更します。詳細についてはflexiportの説明書を参照してください。 \\
    start\_angle & スキャンし始める角度です。ラジアンで指定してください。全スキャンのためにデフォルトで始めるために0に設定してください。 \\
    end\_angle & スキャンしやめる角度です。ラジアンで指定してください。全スキャンのためにデフォルトでやめるために0に設定してください。 \\
    cluster\_count & クラスタサイズです。デフォルトは1です。 \\
    enable\_intensity & Trueに設定された場合、インテンシティデータも送られます。 \\
    high\_sensitivity & いくつかのモデルは高感度モードを特色します。それを可能にするためにこのパラメータを設定してください。 \\
    pull\_mode & プルモードに変わります。サービスポートを通ってスキャンを要求必要になります。 \\
    new\_data\_mode & Trueに設定された場合、毎スキャンは新しいデータを要求します。そうではなければ、最も最新のデータが送られます。 \\
    error\_time & 数秒におけるエラー間の最小の時間です。エラーがこの値より接近してともに生じれば、コンポーネントはエラーに移動します。そうでなければ、レーザーをリセットし、継続することを試みます。 \\
    x, y, z & レーザーの3次元位置です。 \\
    roll, pitch, yaw & レーザーの3次元角度です。 \\
    \bottomrule
  \end{tabularx}
  \caption{Available configuration parameters.}
  \label{tab:config_params}
\end{table}

\section{ポート}
\label{sec:port}

コンポーネントによって提供されるポートはテーブル~\ref{tab:ports}に述べられています。

北洋電気のレーザープロトコルは、20未満の値を使用して、データ内でエラーを報告します。
コンポーネントはこれらの値を0mに取り替えます。データを処理する場合、未知のこととしてゼロの値を
すべて考慮してください。

\begin{table}[t]
  \centering
  \begin{tabularx}{\columnwidth}{lllX}
    \toprule
    ポート名 & ポート型 & データ型 & 意味 \\
    \midrule
    ranges & OutPort & RTC::RangeData & レンジデータ \\
    intensities & OutPort & RTC::IntensityData & インテンシティデータ。インテンシティデータが可能になる場合利用のみです。 \\
    ranger & Service & Ranger & RTC:Rangerインターフェースを提供しているサービスポートです。 \\
    & & HokuyoAist & HokuyoAistインターフェースを提供しているサービスポートです。 \\
    \bottomrule
  \end{tabularx}
  \caption{コンポーネントのポート。}
  \label{tab:ports}
\end{table}

\section{例}
\label{sec:examples}

例のrtc.confファイルは\verb|${prefix}/share/rtc_hokuyoaist/examples/conf/|
フォルダにインストールされています。

\section{Changelog}

\subsection{2.0}

\begin{itemize}
  \item Support hokuyo\_aist library v2 API.
\end{itemize}

\end{Japanese}
\end{CJK}
\end{document}
